%!TEX root = forallxadl.tex
\chapter[Quick reference]{Quick reference}\label{ch.qr}
%\pagestyle{plain}
\section*{Schematic Truth Tables} ~\\
\label{app.SchematicTTs}
\begin{minipage}{0.2\textwidth}
	\begin{tabular}{c|c} \toprule 
\meta{A} & \enot\meta{A}\\
\midrule
T & F\\
F & T \\
\bottomrule
\end{tabular}
\end{minipage}\qquad\begin{minipage}{0.75\textwidth}
	\begin{tabular}{c c|c|c|c|c} \toprule 
\meta{A} & \meta{B} & $\meta{A}\eand\meta{B}$ & $\meta{A}\eor\meta{B}$ & $\meta{A}\eif\meta{B}$ & $\meta{A}\eiff\meta{B}$\\
\midrule
T & T & T & T & T & T\\
T & F & F & T & F & F\\
F & T & F & T & T & F\\
F & F & F & F & T & T\\\bottomrule
\end{tabular}
\end{minipage}




\section*{Symbolisation}
\label{app.symbolization}

\paragraph{Sentential Connectives}~\\

\hspace{1cm}~\begin{tabular}{rl} \toprule
It is not the case that P & $\enot P$\\
Either P, or Q & $(P \eor Q)$\\
Neither P, nor Q & $\enot(P \eor Q)$\ or \ $(\enot P \eand \enot Q)$\\
Both P, and Q & $(P \eand Q)$\\
If P, then Q & $(P \eif Q)$\\
P only if Q & $(P \eif Q)$\\
P if and only if Q & $(P \eiff Q)$\\
P unless Q & $(P \eor Q)$\\
\bottomrule\end{tabular}

\paragraph{Predicates}~\\

\hspace{1cm}~\begin{tabular}{rl} \toprule
All Fs are Gs & $\forall x(Fx \eif Gx)$\\
Some Fs are Gs & $\exists x(Fx \eand Gx)$\\
Not all Fs are Gs & $\enot\forall x(Fx \eif Gx)$\ or\ $\exists x(Fx \eand \enot Gx)$\\
No Fs are Gs & $\forall x(Fx \eif\enot Gx)$\ or\ $\enot\exists x(Fx \eand Gx)$\\
\bottomrule\end{tabular}

\newpage\paragraph{Identity}~\\

\hspace{1cm}~\begin{tabular}{rl} \toprule
Only c is G & $\forall x(Gx \eiff x=c)$\\
Everything besides c is G & $\forall x(\enot x = c \eif Gx)$\\
%$j$ is more $R$ than anyone else. & $\forall x(x≠ j \eif Rjx)$\\
The F is G & $\exists x(Fx \eand \forall y(Fy \eif x=y) \eand Gx)$\\
It is not the case that the F is G & $\enot\exists x(Fx \eand \forall y(Fy \eif x=y) \eand Gx)$\\
The F is non-G & $\exists x(Fx \eand \forall y(Fy \eif x=y) \eand \enot Gx)$\\\bottomrule
\end{tabular}

\section*{Using identity to symbolize quantities}

\subsection*{There are at least \blank\ Fs.}
\label{summary.atleast}

\begin{ekey}
\item[\text{one}] $\exists xFx$
\item[\text{two}] $\exists x_1\exists x_2(Fx_1 \eand Fx_2 \eand \enot x_1  = x_2)$
\item[\text{three}] $\exists x_1\exists x_2\exists x_3(Fx_1 \eand Fx_2 \eand Fx_3 \eand \enot x_1 = x_2 \eand\enot x_1 = x_3 \eand \enot x_2 = x_3)$
% \item[\text{four}] $\exists x_1\exists x_2\exists x_3\exists x_4 (Fx_1 \eand Fx_2 \eand Fx_3 \eand Fx_4 \eand \phantom{x}$\\
%\phantom{$\exists x_1\exists x_2$}$\enot x_1 = x_2 \eand \enot x_1 = x_3 \eand \enot x_1 = x_4 \eand \enot x_2 = x_3 \eand \enot x_2 = x_4 \eand \enot x_3 = x_4)$
\item[n] $\exists x_1…\exists x_n(Fx_1 \eand …\eand Fx_n \eand \enot x_1 = x_2 \eand …\eand \enot x_{n-1} = x_n)$ 
\end{ekey}

\subsection*{There are at most \blank\ Fs.}
\label{summary.atmost}

One way to say `there are at most $n$ Fs' is to put a negation sign in front of the symbolisation for `there are at least $n+1$ Fs'. Equivalently, we can offer:
\begin{ekey}
\item[\text{one}] $\forall x_1\forall x_2\bigl( (Fx_1 \eand Fx_2) \eif x_1=x_2\bigr)$
\item[\text{two}] $\forall x_1\forall x_2\forall x_3\bigl( (Fx_1 \eand Fx_2 \eand Fx_3) \eif (x_1=x_2 \eor x_1=x_3 \eor x_2=x_3) \bigr)$
\item[\text{three}] $\begin{multlined}[t]
	\forall x_1\forall x_2\forall x_3\forall x_4\bigl( (Fx_1 \eand Fx_2 \eand Fx_3 \eand Fx_4) \eif \\
(x_1=x_2 \eor x_1=x_3 \eor x_1=x_4 \eor x_2=x_3 \eor x_2=x_4 \eor x_3=x_4) \bigr)
\end{multlined}$
\item[n]$\forall x_1…\forall x_{n+1}
\bigl( (Fx_1\eand … \eand Fx_{n+1}) \eif (x_1=x_2 \eor … \eor x_n=x_{n+1}) \bigr)$ 
\end{ekey}
% \newpage
\subsection*{There are exactly \blank\ Fs.}
\label{summary.exactly}

One way to say `there are exactly $n$ Fs' is to conjoin two of the symbolizations above and say `there are at least $n$ Fs and there are at most $n$ Fs.' The following equivalent formulae are shorter:
\begin{ekey}
\item[\text{zero}] $\forall x\enot Fx$
\item[\text{one}] $\exists x\bigl(Fx \eand \forall y(Fy \eif x= y) \bigr)$
\item[\text{two}] $\exists x_1\exists x_2\bigl(Fx_1 \eand Fx_2 \eand \enot x_1 = x_2 \eand \forall y\bigl(Fy \eif (y= x_1 \eor y = x_2) \bigr) \bigr)$
\item[\text{three}] $\begin{multlined}[t]
	\exists x_1\exists x_2\exists x_3\bigl( Fx_1 \eand Fx_2 \eand Fx_3 \eand \enot x_1 =  x_2 \eand \enot  x_1 = x_3 \eand \enot x_2 = x_3 \eand \phantom{x} \\
\forall y\bigl( Fy \eif (y = x_1 \eor y = x_2 \eor y =  x_3) \bigr) \bigr)
\end{multlined}$
\item[n] $\begin{multlined}[t]
	\exists x_1…\exists x_n\bigl( Fx_1 \eand …\eand Fx_n  \eand \enot x_1 = x_2 \eand …\eand \enot x_{n-1}= x_n \eand \phantom{x} \\
\forall y\bigl( Fy \eif (y= x_1 \eor … \eor y= x_n) \bigr)\bigr)
\end{multlined}$ 
%\item[one] $\exists x\forall y\bigl(Fx \eand (Fy \eif y = x) \bigr)$
%\item[two] $\exists x\exists y\forall z\Bigl(Fx \eand Fy \eand \bigl(Fz \eif (z=x \eor z=y) \bigr) \eand x ≠ y\Bigr)$
%\item[three] $\exists x_1\exists x_2\exists x_3\forall y\Bigl(Fx_1 \eand Fx_2 \eand Fx_3 \eand \bigl(Fy \eif (y=x_1 \eor y=x_2 \eor y=x_3) \bigr) \eand x_1 ≠ x_2 \eand x_1 ≠ x_3 \eand x_2 ≠ x_3\Bigr)$
%\item[n] $\exists x_1\cdots\exists x_n\forall y\Bigl(Fx_1 \eand\cdots\eand Fx_n \eand \bigl(Fy \eif (y=x_1 \eor \cdots \eor y=x_n) \bigr) \eand x_1 ≠ x_2 \eand\cdots\eand x_{n-1}≠ x_n\Bigr)$ 
\end{ekey}

\newpage
\label{ProofRules}
\section*{Basic deduction rules for \TFL}
\renewenvironment{proof}
	{\noindent\par\noindent\small$\begin{nd}}
	{\end{nd}$\noindent\normalsize\ignorespacesafterend}

%{\LARGE \textbf{Basic Rules of Proof}}
\begin{multicols}{2}


\subsection{Conjunction Introduction, p.\ \pageref{conjint}}
\begin{proof}
	\have[m]{a}{\meta{A}}
	\have[n]{b}{\meta{B}}
	\have[\ ]{c}{\meta{A}\eand\meta{B}} \ai{a, b}
\end{proof}

\subsection{Conjunction Elimination, p.\ \pageref{conjelim}}
\begin{proof}
	\have[m]{ab}{\meta{A}\eand\meta{B}}
\\	\have[\ ]{a}{\meta{A}} \ae{ab}

	\have[m]{ab}{\meta{A}\eand\meta{B}}
\\	\have[\ ]{b}{\meta{B}} \ae{ab}
\end{proof}



\subsection{Conditional Introduction, p.\ \pageref{condint}}
\begin{proof}
	\open
		\hypo[i]{a}{\meta{A}}
		\have[j]{b}{\meta{B}}
	\close
	\have[\ ]{ab}{\meta{A}\eif\meta{B}}\ci{a-b}
\end{proof}

\subsection{Conditional Elimination, p.\ \pageref{condelim}}
\begin{proof}
	\have[m]{ab}{\meta{A}\eif\meta{B}}
\\	\have[n]{a}{\meta{A}}
	\have[\ ]{b}{\meta{B}} \ce{ab,a}
\end{proof}



\subsection{Negation Introduction, p.\ \pageref{negint}}

\begin{proof}
\open
	\hypo[i]{a}{\meta{A}}
	\have[j]{b}{\meta{B}}
	\have[k]{nb}{\enot\meta{B}}
\close
\have[\ ]{na}{\enot\meta{A}}\ni{a-b,a-nb}
\end{proof}


\subsection{Negation Elimination, p.\ \pageref{negelim}}
\begin{proof}
\open
	\hypo[i]{a}{\neg\meta{A}}
	\have[j]{b}{\meta{B}}
	\have[k]{nb}{\enot\meta{B}}
\close
\have[\ ]{na}{\meta{A}}\ne{a-b,a-nb}
\end{proof}





\subsection{Disjunction Introduction, p.\ \pageref{disjint}}

\begin{proof}
	\have[m]{a}{\meta{A}}
	\have[\ ]{ab}{\meta{A}\eor\meta{B}}\oi{a}

	\have[m]{a}{\meta{A}}
\\	\have[\ ]{ba}{\meta{B}\eor\meta{A}}\oi{a}
\end{proof}

\subsection{Disjunction Elimination, p.\ \pageref{disjelim}}
\begin{proof}
	\have[m]{ab}{\meta{A}\eor\meta{B}}
\\	\open
		\hypo[i]{a}{\meta{A}}
		\have[j]{c1}{\meta{C}}
	\close
	\open
		\hypo[k]{b}{\meta{B}}
		\have[l]{c2}{\meta{C}}
	\close
	\have[\ ]{c}{\meta{C}} \oe{ab,a-c1, b-c2}
\end{proof}



\subsection{Biconditional Introduction, p.\ \pageref{biint}}
\begin{proof}
	\open
		\hypo[i]{a1}{\meta{A}} 
		\have[j]{b1}{\meta{B}}
	\close
	\open
		\hypo[k]{b2}{\meta{B}}
		\have[l]{a2}{\meta{A}}
	\close
	\have[\ ]{ab}{\meta{A}\eiff\meta{B}}\bi{a1-b1,b2-a2}
\end{proof}

\subsection{Biconditional Elimination, p.\ \pageref{bielim}}
\begin{proof}
	\have[m]{ab}{\meta{A}\eiff\meta{B}}
\\	\have[n]{a}{\meta{A}}
	\have[\ ]{b}{\meta{B}} \be{ab,a}

	\have[m]{ab}{\meta{A}\eiff\meta{B}}
\\	\have[n]{a}{\meta{B}}
	\have[\ ]{b}{\meta{A}} \be{ab,a}
\end{proof}

\subsection{Reiteration, p.\ \pageref{reit}}

\begin{proof}
	\have[m]{a}{\meta{A}}
	\have[\ ]{}{\vdots}
	\have[\ ]{c}{\meta{A}} \by{R}{a}
\end{proof}


\end{multicols}


\section*{Derived rules for \TFL, §\ref{s:Derived}}
\begin{multicols}{2}

\subsection*{Disjunctive syllogism}
\begin{proof}
	\have[m]{ab}{\meta{A} \eor \meta{B}}
	\have[n]{nb}{\enot \meta{A}}
	\have[\ ]{con}{\meta{B}}\by{DS}{ab, nb}

	\have[m]{ab}{\meta{A} \eor \meta{B}}
\\	\have[n]{nb}{\enot \meta{B}}
	\have[\ ]{con}{\meta{A}}\by{DS}{ab, nb}
\end{proof}



\subsection*{Modus Tollens}

\begin{proof}
	\have[m]{ab}{\meta{A}\eif\meta{B}}
	\have[n]{a}{\enot\meta{B}}
	\have[\ ]{b}{\enot\meta{A}} \by{MT}{ab,a}
\end{proof}

\subsection*{Double Negation Elimination}


	\begin{proof}
		\have[m]{dna}{\enot \enot \meta{A}}
		\have[ ]{a}{\meta{A}}\dne{dna}
	\end{proof}

\subsection*{Tertium non datur}
	\begin{proof}
		\open
			\hypo[i]{a}{\meta{A}}
			\have[j]{c1}{\meta{B}}
		\close
		\open
			\hypo[k]{b}{\enot\meta{A}}
			\have[l]{c2}{\meta{B}}
		\close
		\have[\ ]{ab}{\meta{B}}\tnd{a-c1,b-c2}
	\end{proof}


	
%
%\subsection*{Hypothetical Syllogism}
%
%\begin{proof}
%	\have[m]{ab}{\meta{A}\eif\meta{B}}
%	\have[n]{bc}{\meta{B}\eif\meta{C}}
%	\have[\ ]{ac}{\meta{A}\eif\meta{C}}\by{HS}{ab,bc}
%\end{proof}

\subsection*{De Morgan Rules}
\begin{proof}
	\have[m]{ab}{\enot (\meta{A} \eor \meta{B})}
	\have[\ ]{dm}{\enot \meta{A} \eand \enot \meta{B}}\dem{ab}

	\have[m]{ab}{\enot \meta{A} \eand \enot \meta{B}}
\\	\have[\ ]{dm}{\enot (\meta{A} \eor \meta{B})}\dem{ab}

	\have[m]{ab}{\enot (\meta{A} \eand \meta{B})}
\\	\have[\ ]{dm}{\enot \meta{A} \eor \enot \meta{B}}\dem{ab}

	\have[m]{ab}{\enot \meta{A} \eor \enot \meta{B}}
\\	\have[\ ]{dm}{\enot (\meta{A} \eand \meta{B})}\dem{ab}
\end{proof}
\end{multicols}


\newpage
\section*{Basic deduction rules for \FOL}

\begin{multicols}{2}
\subsection*{Universal elimination, p.\ \pageref{unielim}}

\begin{proof}
	\have[m]{a}{\forall \meta{x}\meta{A}}
	\have[\ ]{c}{\meta{A}\subs{\meta{c}}{\meta{x}}} \Ae{a}
\end{proof}
\noindent 	\meta{c} can be any name

\subsection*{Universal introduction, p.\ \pageref{uniint}}

\begin{proof}
		\have[m]{a}{\meta{A}\subs{\meta{c}}{\meta{x}}}
	\have[\ ]{c}{\forall \meta{x}\meta{A}} \Ai{a}
\end{proof}

\noindent 	\meta{c} must not occur in any undischarged assumption, or in \meta{A}

\subsection*{Existential introduction, p.\ \pageref{exint}}

\begin{proof}
	\have[m]{c}{\meta{A}\subs{\meta{c}}{\meta{x}}} 
			\have[\ ]{a}{\exists \meta{x}\meta{A}} \Ei{c}
\end{proof}
\noindent 	\meta{c} can be any name

\subsection*{Existential elimination, p.\ \pageref{exelim}}

\begin{proof}
	\have[m]{a}{\exists \meta{x}\meta{A}}
	\open	
		\hypo[i]{b}{\meta{A}\subs{\meta{c}}{\meta{x}}}
		\have[j]{c}{\meta{B}}
	\close
	\have[\ ]{d}{\meta{B}} \Ee{a,b-c}
\end{proof}

\noindent \meta{c} must not occur in any undischarged assumption, in $\exists \meta{x}\meta{A}$, or in \meta{B}\vfill\columnbreak
\end{multicols}
\begin{multicols}{2}
\subsection*{Identity introduction, p.\ \pageref{idint}}

\begin{proof}
	\have[\ \,\,\,]{x}{\meta{c}=\meta{c}} \idi{}
\end{proof}


\subsection*{Identity elimination, p.\ \pageref{idelim}}

\begin{proof}
	\have[m]{e}{\meta{a}=\meta{b}}
	\have[n]{a}{\meta{A}\subs{\meta{a}}{\meta{x}}}
	\have[\ ]{ea1}{\meta{A}\subs{\meta{b}}{\meta{x}}} \ide{e,a}
\end{proof}
\end{multicols}

\section*{Derived rules for \FOL, §\ref{s:CQ}}
\begin{multicols}{2}
\begin{proof}
	\have[m]{ab}{\forall \meta{x}\enot \meta{A}}
	\have[\ ]{ac}{\enot \exists \meta{x} \meta{A}}\by{CQ$_{\forall/\enot\exists}$}{ab}

	\have[m]{ab}{\enot \exists \meta{x}  \meta{A}}
\\	\have[\ ]{ac}{\forall \meta{x}\enot\meta{A}}\by{CQ$_{\enot\exists/\forall}$}{ab}
\end{proof}
\begin{proof}
	\have[m]{ab}{\exists \meta{x}\enot\meta{A}}
	\have[\ ]{ac}{\enot \forall \meta{x} \meta{A}}\by{CQ$_{\exists/\enot\forall}$}{ab}

	\have[m]{ab}{\enot \forall \meta{x}  \meta{A}}
\\	\have[\ ]{ac}{\exists \meta{x}\enot \meta{A}}\by{CQ$_{\enot\forall/\exists}$}{ab}
\end{proof}

\subsection*{Alternative identity elimination, p.\ \pageref{id.es}}
\begin{proof}
		\have[m]{e}{\meta{a}=\meta{b}}
	\have[n]{a}{\meta{A}\subs{\meta{b}}{\meta{x}}}
	\have[\ ]{ea1}{\meta{A}\subs{\meta{a}}{\meta{x}}} \by{=ES}{e,a}
	\end{proof}
\end{multicols}
